This policy document has been prepared for the guidance of everyone working as
part of the volunteer team at EMF and follows Home Office and Department of
Health recommendations. It is essential that all team members adhere to these
guidelines. In the event of a query, team members are advised to consult the
team co-ordinator or her assigned deputy or the appropriate shift leader for
further guidance.

These guidelines are intended as a practical framework for people working with
children in voluntary settings to help ensure the safety, well-being and
protection of children in their charge.

It is the responsibility of every member of the EMF volunteer teams to prevent
the physical, sexual and psychological abuse or neglect of children and young
people, or vulnerable adults, in our care and to report any such abuse that may
be suspected or discovered.

The Lost and Found Children service will be provided 24 hours a day while
ticket holders are on site. All enquiries and dealings regarding lost and found
children will be co-ordinated by the EMF First Aid Team and all staff on site
will be briefed about this.

The EMF First Aid Tent is the designated lost child point and will be marked as
such on any maps, printed or online EMF information.

\subsection{Reporting Protocols}

Upon receiving a report of a missing or found child, young person or vulnerable
adult, staff will notify HQ as soon as practicable. HQ will forward this
information on to the first aid team, either via radio (between the hours of
8am and midnight) or via mobile phone (between the hours of 2am and 10am).

All staff at EMF should be made aware as soon as possible, noting the
caveat concerning radios below. All staff on gates to the site should not allow
any child to leave the site until it has been confirmed with the First Aid Team
that the child is not reported lost. Announcements should be made at each
stage. These announcements will be treated as a priority and will be broadcast
at the earliest opportunity. Announcements will not refer to the child
specifically or give personal details, descriptions or names.

Found children should be reported to HQ in a similar manner. In addition, upon
finding a lost child or vulnerable adult, the volunteer in question should make
immediate steps to bring another volunteer to the scene as quickly as
practicable, if they are on their own. It is essential that a lost child or
vulnerable adult not be left in the care of one person. A pair of first-aiders
from the will be dispatched and visit the scene in order to escort the child or
vulnerable adult to the EMF First Aid Tent.

While in the care of the EMF First Aid Team, every effort will be made to
ensure the comfort, safety and well-being of the child, young person or
vulnerable adult in a manner which does not violate their human rights and
follows the recommended guidance. Efforts will be made to re-unite the
individual with their parent or guardian, as appropriate, or referral to
statutory agencies as appropriate.

It should be noted that the EMF First Aid Team has no right to detain any
person -- child or considered vulnerable adult -- against his or her wishes.
Efforts will be made to negotiate the best course of action for that
individual.

If there is any suspicion of abuse or neglect of the child or vulnerable adult,
the EMF First Aid Team Leader or Deputy Team Leader must be informed and a
decision will be taken whether to involve the relevant services, such as the
Police \& Social Services.

Time scales will be taken into consideration. If a child or vulnerable adult is
not found within a reasonable time, or a found child is not re-united with a
parent or guardian within a certain time, local authorities will be contacted,
and the situation escalated.

Any individual who is behaving, or expressing a serious intention to behave, in
a manner likely to harm themself or others should be considered at risk.
Support from security and/or Police may be needed while the situation is
assessed.

Any parent/guardian of a child or young person, or friend of a missing person,
who reports them missing may need support and it is to be expected that the
member of EMF staff will direct them, or escort if necessary, to the EMF First
Aid Tent. They may be considerably distressed. At this point, staff should keep
details minimal when notifying the EMF First Aid Team; the team will take full
details.

When a child is reunited with their parent or guardian, identification should be
requested and recorded. Only in extreme circumstances should a child be
allowed to leave without the parent providing some form of ID. Should the child
seem afraid or unwilling to accompany the parent or guardian then assistance
from the Police should be sought. Equally, should the parent or guardian
seem in any way unfit to care for that child then assistance from the
Police may be sought.

\subsection{Radio Usage}
All efforts will be made to restrict the amount of information given over the
radio, such as names or other identifying details. A fixed or mobile phone line
should be used wherever possible. Radio codewords for children will be in
use at this event, and all staff with radios will be briefed on these.

\subsection{Definitions and Key Terms}

The Children's Act (1989) defines a child as any person under the age of 18
years. For practical considerations at events such as this, each young person
will be assessed on a case by case basis with regards to the safely and well
being of a minor.

The definition of a vulnerable adult is given in the `No Secrets' guidelines
published by the Department of Health in 2000 as someone ``who is or may be in
need of community care services by reason of mental or other disability, of age
or illness; and who is or may be unable to take care of him or herself, or
unable to protect him or herself against significant harm or exploitation''
Further, it defines abuse as ``a violation of an individual's human and civil
rights by any other person(s).''

In particular, it should be noted here that adults (i.e. those aged 18 years or
higher) have the right to make their own decisions unless there are clear
grounds to override this as a result of their lack of capacity OR if wider
public interest is involved.

The law in relation to adults offers far fewer opportunities or
responsibilities in relation to intervention. The principle here is to promote
negotiation with regard to the individual's capacity at that time.

It is essential that the boundaries of confidentiality are explained to the
child or young person or vulnerable adult - if possible before disclosure, i.e.
where it is suspected they might be about to disclose. Under the Children's Act
(1989), we have a duty to inform Social Services of any reports of abuse
involving children and cannot therefore keep such details confidential. This is
for the protection of the individual and possibly others. It is the role of the
team co-ordinator or her assigned assistant to liaise with Social Services in
this matter and she is responsible for making them aware of the disclosure.

Written notes will be kept of all relevant information. Information should
however only be shared on a strictly need-to-know basis.
