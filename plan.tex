\section{Event Overview}

Electromagnetic Field (EMF) is the UK's largest technology-focused camping festival:
A three-day event for people with an inquisitive mind and an interest in science, engineering,
technology, crafts, DIY, and computer security.

EMF can be seen as a cross between a conference and a music festival, with talks and workshops on a wide
range of subjects. In addition, there will be demonstrations and installations by attending
members of the community, and a small amount of live music.

EMF is a non-profit event run by a dedicated team of volunteers who have
experience staffing and organising events for the European maker community.

\subsection{Key Information}

\begin{tabular}{l l}
Public Event Hours & 08:00 Friday 29\th August -- 17:00 Monday 1\st September 2014 \\
Site Capacity & 1500 \\
Ticket Price & £90--£120 \\
Location & Hounslow Hall Estate, Drayton Road, Newton Longville, MK17 0BU \\
\end{tabular}

\subsection{Context}

EMF 2014 is the third event, and the second major camping festival, organised by
Electromagnetic Field Ltd.

EMF follows in the footsteps of a number of similar events across Europe and the USA,
the most recent being the successful 3500-capacity Observe Hack Make festival last year
in the Netherlands.

Our last three-day festival in 2012 brought more than 600 attendees to Pineham Park in
Milton Keynes for talks and workshops on topics ranging from genetic modification to electronics,
blacksmithing to high-energy physics, reverse engineering to lock picking,
computer security to crocheting, and quadcopters to beer brewing.

\subsection{Demographic}

EMF events have historically attracted a broad spectrum of attendees due to the variety of talks
and workshops available.

The audience demographic is expected to broadly range in age between 18 and 50, with the majority
of attendees being between 22 and 35, with a slight male bias.

\section{Management}

Electromagnetic Field is an entirely volunteer-run event, but we strive to operate the event to
a standard equivalent to professionally-run events. A number of members of the organising team have
significant experience running similar events across Europe, including the 3500-capacity
Observe Hack Make festival in 2013.

Overall responsibility for the operation of the event lies with the Directors of Electromagnetic Field, Russell Garrett and Jonty Wareing.

Those volunteers involved in the running event are formed into teams, with each team having an experienced lead and a deputy lead who are accountable for that team.

Organisational meetings of all team leads are held periodically prior to the event. During the event, a senior member of the event management will be on call as duty site manager to respond to incidents. Meetings of team leads will be held daily during the event.

\section{Site}

The event will be located in farmland at the Hounslow Hall Estate, Newton Longville.
The site is operated by Explore The Country and EMF has a license to use it between
Monday 25\superscript{th} August and Thursday 4\superscript{th} September.

Where not otherwise secure, the site will be surrounded by a Heras-style fence for security purposes.
The secure licensing perimeter will surround the event tents and all camping areas.

The entrance gate will be staffed 24 hours per day, and tickets will be
exchanged for wristbands on entry.

Further detail of the site layout may be found on the attached site plan.

\section{Licensing}

EMF will be covered by a Premises License for the sale of alcohol
and the provision of regulated entertainment. The proposed licensed hours are:

\begin{tabular}{l l}
Friday 29\th August & 1200 -- 0200 \\
Saturday 30\th August & 1200 -- 0200 \\
Sunday 31\st August & 1200 -- 0100 \\
\end{tabular}

\subsection{Alcohol Sales}

The bar will be operated by EMF and staffed by volunteers. All bar staff will receive a
briefing on legislation and event policies before starting work. The bar will operate a
``Challenge 21'' policy for dealing with under-18s, and will only accept
approved documents as proof of age. Bar staff will be instructed not to serve customers who are
drunk, and will be familiarised with the strength of the drinks they are serving.

A full price list will be provided at each bar, which will include the ABV levels of each drink
and the measured quantity in which spirits are being sold.

Drinks will not be served in glass containers. Attendees will be advised not to bring
glass onto the site.

\subsection{Regulated Entertainment}

The main focus of EMF is talks and workshops, however we will provide some live music for evening
entertainment, as well as ancillary recorded music between talks. Music is a secondary focus of the
event, and will not be a major component of any promotion or advertising.

\subsection{Public Nuisance}

The site is not in direct proximity to any residential areas, and is located in a
natural dip in the land. As EMF is a camping festival, attendees are not expected to leave the site
at night, so there is a low risk of attendees causing public nuisance outside the site.

\section{Noise}

We are mindful of the need to keep noise nuisance to an absolute minimum and EMF will cooperate fully
with Environmental Health and local residents to achieve this.

Staff involved with noise monitoring, the site manager, and staff at any locations within the site
using amplified music will be in radio contact and instructed to effectively reduce noise levels if
necessary. Noise levels will be checked at least once per day whenever there is live music being played.

No amplified music will be permitted on site outside of the licensed hours.

\subsection{Control}

In order to manage the noise impact on the local area, EMF will implement the following noise control plan.

As agreed with Environmental Health, EMF will ensure noise output remains below the following thresholds,
measured over a 5 minute period, at noise sensitive boundaries:

\begin{tabular}{l l}
  Before 2300 & 15dB(A) above ambient\\
  After 2300 & 10dB(A) above ambient\\
\end{tabular}

In addition, EMF will ensure that the noise level in the frequency band 63--125Hz will not exceed 70dB(L)
at noise sensitive boundaries.

\section{Children}

The event will be family-friendly with a dedicated children's area. Under-12s will receive free tickets
and under-18s will receive a discount.

An DBS-checked volunteer will always be on-call in case of a lost child situation.

\section{Toilets \& Sanitation}

As there are no toilet facilities on site, a minimum of 20 toilets and 5 urinals, as well as two disabled toilets,
will be provided. This is well in excess of the recommendations made by the HSE Event Safety Guide.

Mains water is available on site and taps and washbasins will be provided. All temporary water installation to be
handled by a contractor in compliance with Water Authority Regulations.

Showers will also be provided.

\section{Food}

Food on site will be provided by commercial food concessions. Food hygeine certificates will be checked and kept on file for all food vendors.

\section{Staffing}

As is common with similar events, we aim to provide as many staff as possible
by asking attendees to volunteer. All stewarding will be overseen by an experienced stewarding co-ordinator.

All key members of staff will be issued with a radio and trained in its use. A ``control'' member of staff will be
contactable by radio at all times and will have emergency contact details for the organisation team.

\subsection{Staffing Levels}

Staff levels will be allocated as follows:

\begin{tabular}{l l l}
Role & Period & Staff \\
\hline
Main Gate & 24/7 & 2 -- 4 \\
Information Point & 24/7 & \\
Roving & 24/7 & 2 -- 4 \\
Bar & Licensed Hours & 2 -- 5 \\
Per Tent & During Talks & 2 \\
Control & 24/7 & 1
\end{tabular}

As well as these staff, a dedicated duty event manager will be on call at all times during the event.

In the unlikely event staffing levels cannot be guaranteed, external stewarding services will be sought.

\section{First Aid}

First aid will be provided by our own trained volunteers, including those with advanced St John's Ambulance
qualifications and previous experience in large-scale festival first aid.

During setup and teardown there will be at least one qualified first-aider on duty.

During the event, there will be at least two qualified first aiders are on duty at all times.

\section{Transport and parking}

Attendees will be encouraged to use public transport as much as possible. Car parking on site will be ticketed,
and vehicles will not be allowed on site without a pass.

Around 900 vehicles are expected to park on site, and the capacity of the car parking fields is in excess of 1000.
The car parking fields are located some distance away from the festival site, and a shuttle service will run between
the two locations.

A scheduled shuttle bus service will be provided between the site and Bletchley station.

\subsection{Traffic Management}

The sole means of access to the site is from a single access road off Drayton Road. While cars are arriving, stewards will be posted to marshall the traffic.

There is a gate which controls access to the site, which will be closed during off-peak hours. The gate can be remotely operated by staff if vehicles need to be allowed onto site.

Procedures will be in place to quickly clear the access road of traffic by running vehicles off into adjacent fields if emergency services require access to the site. Stewards handling traffic management will be briefed on these procedures.

\section{Fire risk}
\subsection{Sources of ignition}

The main sources of ignition at EMF are:

\begin{itemize}
\item Hot exhaust from generators
\item LPG appliances in catering area
\item Camp fires and gas appliances used by attendees
\end{itemize}

\subsection{Steps to minimise risk}
The following steps will be taken to mitigate risks of fire:

\begin{itemize}
\item Generators provided by EMF will be sited away from all combustible materials in accordance with supplier's guidance.
\item No other generators will be allowed on site.
\item Combustible materials will be stored away from structures.
\item Firefighting equipment will be provided on site and make staff aware of its location. Fire extinguishers will be sited in all event tents.
\item Fire lanes will be provisioned within camping areas and clearly marked.
\item Camp fires will not be allowed on the ground.
\item Attendees will be instructed not to use gas appliances in tents.
\item Fire lanes will provided in camp sites and monitored to be kept clear.
\item Roving staff will be instructed to monitor the site for any fire hazards and contact control over radio.
\item Catering/concessions staff will be made aware of regulations regarding gas storage.
\item Catering area will be sited well away from camping area.
\item Sufficient access to the site will be provided and maintained clear for access of fire appliances.
\item Weather conditions will be monitored in case of very dry conditions raising the risk of spread of fire through vegetation.
\end{itemize}

\subsection{Emergency plan}

All stewards will be briefed on steps to take if a fire is discovered which will include alerting other staff by radio and, if necessary, evacuating attendees.

\begin{landscape}
\section{Overall Risk Assessment}
\begin{tabular}{| p{3cm} | l | p{1.5cm} | p{8cm} | p{1.5cm} | p{2cm} | p{5cm} |}
\hline
\textbf{Hazard} & \textbf{Risk} & \textbf{Affected Parties}
& \textbf{Control Measures} & \textbf{Resulting Risk} & \textbf{Responsible Team} & \textbf{Comments} \\ \hline

Electrocution & Moderate & Everyone &
All electrical installations to conform to BS7671. 30mA RCDs on all circuits.
Regular visual checks. Attendees who require power should be briefed on the risks. &
Moderate & Power & \\ \hline

Fire & Moderate & Everyone & Please refer to previous section on fire risks. &
Low & All & \\ \hline

Injury from vehicles operating on site & Moderate & Everyone &
Vehicle movements on site to be restricted during peak hours
(11:00--23:00) and co-ordinated by radio. No un-marshalled vehicles during peak hours. &
Low & Stewards & \\ \hline

Trips \& Falls & Moderate & Everyone &
As far as is practical, ensure all cables are buried or flown above head height. Ensure site is adequately lit. &
Moderate & Setup & Trip hazards (guy ropes, etc.) will always be present on a camp site.\\ \hline

Glass injuries & Moderate & Everyone &
Discourage bringing glass onto site. Alcohol should be served in plastic or paper cups. &
Low & Stewards & \\ \hline

Drowning in Lake & Moderate & Everyone & Lake to be well-lit and adequately fenced off with
pedestrian barriers. & Low & Setup & \\ \hline

Public order issues & Low & Everyone &
Stewards to monitor situation and report by radio. &
Low & Stewards & Event is expected to be low-energy. \\ \hline

Injury from temporary structures & Low & Everyone &
Reputable contractors should be used. &
Low & Setup & \\ \hline

Dehydration \& Sunburn & Low & Everyone &
Water readily available. First aiders on site. & Low & First Aid & \\ \hline

Insect bites \& stings & Low & Everyone &
First aiders on site. & Low & First Aid & \\ \hline

\end{tabular}

\end{landscape}


\section{Event Contact}

Russ Garrett (Event Manager \& Designated Premises Supervisor)

2 Ockendon Mews \\
London N1 3JL

Email: russ@emfcamp.org \\
Telephone: 07799 027 946 (email contact preferred, unless urgent)

